\section{Hazard identification}
\label{chapter4}

\subsection{Identified hazards and countermeasures}


	\begin{itemize}
		\item \label{hazc.1} \textbf{Hazard 1:}
            \\	Co2 sensor delivers incorrect values \ref{req.2} \\\\
            \textbf{Countermeasures 1:}\\
            All measured values are compared with the average of the previously measured values. If a certain threshold for this difference is exceeded, the value is declared invalid.
		\item \label{hazc.2} \textbf{Hazard 2:}
            \\	The window is stuck and does not open or close \ref{req.4} \\\\
            \textbf{Countermeasures 2:}\\
            The position of the stepper motor is measured after the opening or closing process has been completed and compared with the default value of a closed or opened window
		\item \label{hazc.3} \textbf{Hazard 3:}
            \\	Arduino fails \ref{req.1} \\\\
            \textbf{Countermeasures 3:}\\
            The Raspberry Pi is sending an email to the system administrator after connection-retries with the Arduino are unsuccessful for 30 seconds.
	\end{itemize}
	
\subsection{Identified hazards without countermeasures }

	\begin{itemize}
		\item \label{haz.1} Hazard 1: \\
            \\	The Co2 sensor fails and does not provide data anymore.
		\item \label{haz.2} Hazard 2: \\
            \\	The power supply fails on the Arduino or the Raspberry Pi.
		\item \label{haz.3} Hazard 3: \\
            \\	The engine for the window fails. \\
	\end{itemize}