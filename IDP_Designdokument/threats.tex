\section{Threat identification}
\label{chapter3}

\subsection{Identified threats and countermeasures}

	\begin{enumerate}
		\item  \textbf{Threat 1:}
            Someone could manipulate the window this way, intentionally or unintentionally
            that it no longer closes after opening. This could result in a break-in
            consequences. \\ \\         
             \textbf{Mitigation:}
            The device carries out a self-test when commissioning (open 1 x and close 1 x)
		\item \textbf{Threat 2}:
            The window couldn't open because someone put something in front of the window.
            This means there is no ventilation. The system no longer works.\\ \\
            \textbf{Mitigation:}
            Refer to mitigation of Threat 1:
		\item \textbf{Threat 3}:
            DoS: Someone could carry out a DoS attack on the Rasperry PI. This will
            prevents emails from being sent.\\ \\
            \textbf{Mitigation:}
            Installation of a firewall. This means that the Rasperry PI cannot be reached from outside.
	\end{enumerate}
	
\subsection{Identified threats without countermeasures}

	\begin{enumerate}
		\item \textbf{Threat 1:} 
            Someone could try to deliberately trigger the CO 2 sensor. This will open the window
            opened and a break-in could be carried out through the window.  
		\item \textbf{Threat 2:}
            Someone could install a jammer and thus disrupt the connection between Rasperry PI and Arduino and thus, for example, prevent the window from closing
            to carry out a break-in.
		\item \textbf{Threat 3:}
            Someone could install a jammer and prevent the safety circuit from tripping. This means the system no longer works.            
	\end{enumerate}