\section{Threat identification}
\label{chapter3}

\subsection{Identified threats and countermeasures}

	\begin{enumerate}
		\item  \textbf{Threat 1:}
            An attacker can make a log lose or confuse data. \\ \\         
             \textbf{Mitigation:}
            There must be logs with timestamps.
		\item \textbf{Threat 2}:
            An attacker can alter and manipulate the data that is sent back and forth. (Mitm-attack)\\ \\
            \textbf{Mitigation:}
            Each message between the Arduino and Raspberry Pi must be accompanied by an HMAC.
		\item \textbf{Threat 3}:
            An Attacker carry out a DoS attack on the Raspberry Pi.\\ \\
            \textbf{Mitigation:}
            The Raspberry Pi must accept no more than 1 message every 9 seconds from the Arduino to prevent flooding attacks.
            \end{enumerate}
	
\subsection{Identified threats without countermeasures}

	\begin{enumerate}
		\item \textbf{Threat 1:} 
            An attacker can carry out a DoS-attack on the LAN connection of the raspberry pi in order to prevent it from sending an email.
		\item \textbf{Threat 2:}
            An attacker can provide or control state information.
		\item \textbf{Threat 3:}
            An attacker can replay data without detection.            
	\end{enumerate}